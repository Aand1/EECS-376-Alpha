Roslaunch files made splitting our packages into separate nodes easier. 
Instead of having to remember the correct executable s to run and the proper order, 
we created one launch file that incorporated each of our nodes. 
Within each node package there is a directory called launch containing its own launch file called main.launch. 
There is a launch file that finds and runs each main.launch file for each package. 
One issue that we had was the launch of the look\_ahead node. In the simulator, 
look\_ahead subscribes to base\_scan, but on Jinx look\_ahead must subscribe to the topic base\_laser1\_scan. 
In order to avoid having to edit and recompile our code for the module, we simply used the remap command to echo 
the subscriptions from both base\_scan and base\_laser1\_scan. With the remap command it did not matter 
whether we were testing code on the simulator or on Jinx.

While commands from the launch files were able to assist in the versatility of our code, we did run into issues 
with launching files on Jinx. One issue was trying to use the find command to change into a package directory. The 
launch file was not able to locate the package cwru\_semi\_stable, so that we would be able to
 launch cwru\_bringup\_no\_tele.launch. While the roscd command line tool was able to locate the package, 
the find function of ROS Launch XML was not capable of locating the directory. We had to work around this by 
writing a simple script to launch cwru\_bringup\_no\_tele.launch and start\_amcl\_2ndfloor.launch.
While ROS Launch XML is a great framework there could be better documentation and high consistency for ease of use.

