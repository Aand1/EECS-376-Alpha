\section{Steering Node}

The Steering node is responsible for taking the desired velocities from
the velocity planner, the desired path from the path planner, and the
current position to determine correction factors to the velocity command
such that the robot does not stray from its desired path.

\subsection{Theory of Operation}

describe what we want the steering node to do in more exact detail. say
how they interconnect and how we use the messages

\subsubsection{Steering Algorithm}

We will use the linear steering algorithm. This algorithm takes a number
of inputs such as:

\begin{itemize}
\item
  The $x$ and $y$ coordinates of the destination.
\item
  The current $x$ and $y$ coordinates.
\item
  Tuning parameters $K_d$ and $K_\theta$
\end{itemize}
First we use the path segment coordinates to calculate the desired
heading with the following: $atan(yf-ys,xf-xs)$. Next we find a
$d_\theta$ by subtracting $heading_{dest}-heading_{curr}$. We can
prevent turning the long way by checking to see that $d_\theta$ is less
than or greater than $\pi$. Finally, we take the vector components of
the desired heading $tx=cos(heading_{dest})$, $ty=-sin(heading_{dest})$
and dot these with the vectors from the start point to the current point
$xrs*nx+yrs*ny$. We take this product and add it to $d_\theta$ to get
the final corrected heading $-K_d*offset+K_{\theta}*d_\theta$.

\subsection{Observations}

problems we ran into operating observations

\subsection{Coding Procedure}

The \texttt{steering\_example.cpp} sample code was tweaked to include a
file reader to allow us to easily modify constants $K_d$ and $K_\theta$.

\subsection{Future Plans}

\begin{itemize}
\item
  non linear steering
\end{itemize}
