\section{Steering Node}

The Steering node is responsible for taking the desired velocities from
the velocity planner, the desired path from the path planner, and the
current position to determine correction factors to the velocity command
such that the robot does not stray from its desired path.

\subsection{Theory of Operation}

We use a number of messages to control the steering node.

\begin{itemize}
\item
  Path segment: the path segment node passes in the current path segment
  we would like the steering node to follow.
\item
  vel\_des: this message gives us the desired velocity as determined by
  the velocity\_profiler.
\item
  TODO: others?
\end{itemize}
\subsubsection{Steering Algorithm}

We will use the linear steering algorithm. This algorithm takes a number
of inputs such as:

\begin{itemize}
\item
  The $x$ and $y$ coordinates of the destination.
\item
  The current $x$ and $y$ coordinates.
\item
  Tuning parameters $K_d$ and $K_\theta$
\end{itemize}
First we use the path segment coordinates to calculate the desired
heading with the following: $atan(yf-ys,xf-xs)$. Next we find a
$d_\theta$ by subtracting $heading_{dest}-heading_{curr}$. We can
prevent turning the long way by checking to see that $d_\theta$ is less
than or greater than $\pi$. Finally, we take the vector components of
the desired heading $tx=cos(heading_{dest})$, $ty=-sin(heading_{dest})$
and dot these with the vectors from the start point to the current point
$xrs*nx+yrs*ny$. We take this product and add it to $d_\theta$ to get
the final corrected heading $-K_d*offset+K_{\theta}*d_\theta$.

\subsection{Observations}

We ran into several problems when trying to perfect our steering code.
The hardest part of this demo was getting all of the previous nodes
integrated and functioning. We had to integrate several dummy messages
and structures to glue together everything until all the nodes can be
completed.

-TODO: describe bug where the robot would get too close to the door on
the first turn

\subsection{Coding Procedure}

The \texttt{steering\_example.cpp} sample code was tweaked to include a
file reader to allow us to easily modify constants $K_d$ and $K_\theta$.
We also ended up coding a rudimentary path planner that has hard coded
path segments.

\subsection{Future Plans}

\begin{itemize}
\item
  Non-linear steering: We plan on replacing the linear steering
  algorithm with the non-linear one
\item
  Arc path steering: We plan on generating arch path segments and
  allowing the steering node to maintain control over these segments.
\item
  Python source code: We plan on porting all of our existing code to
  python. We would like to utilize rospy for ease of programming and to
  get rid of compile time.
\end{itemize}
